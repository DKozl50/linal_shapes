\documentclass[border=2pt]{standalone}

\usepackage{tikz}
\usepackage{pgfplots,tikz-3dplot}
\pgfplotsset{compat=1.16}

\makeatother

\begin{document}
	
	\pgfplotsset{ % цветовая палитра, параметры внутренней поверхности
		colormap={cmap_in}{
			color(0cm)=(violet!70)
			color(1cm)=(red!20)
		},
	}
	
	\begin{tikzpicture}
	
	\def \xmax {2.2} % константы, отвечают за диапазон осей
	\def \xmin {-2}
	\def \ymax {2.4}
	\def \ymin {-2}
	\def \zmax {4.8}
	\def \zmin {-4.5}
	
	\def \visX {1} % константы, отвечают за диапазон, в котором оси видны мутно (закрыты куском графика)
	\def \visY {2/3}
	\def \visZup {3.05}
	\def \visZdown {-4}
	
	\begin{axis}[
	width=15cm,
	view={145}{20},
	axis equal image,
	axis lines=center,
	xlabel={$x$}, % марки осей
	ylabel={$y$},
	zlabel={$z$},
	xmax=\xmax, xmin=\xmin,
	ymax=\ymax, ymin=\ymin,
	zmax=\zmax, zmin=\zmin,
	ticks=none,
	axis line style={draw=none},
	colormap name=cmap_in,
	mesh/interior colormap={cmap_out}{  % цветовая палитра, параметры внешней поверхности
		color(0cm)=(blue!80),
		color(1 cm)=(cyan!25)},
	]
	
	\draw (axis cs:\xmin, 0, 0) -- (axis cs:\visX, 0, 0); % координатные оси
	\draw (axis cs:0, \ymin, 0) -- (axis cs:0, \visY, 0);
	\draw (axis cs:0, 0, \zmin) -- (axis cs:0, 0, \visZdown);
	
	\addplot3[ % основной график
	fill opacity=0.6, draw opacity=0.7,
	surf,
	shader=flat,
	samples=60,
	domain=0:2*pi,y domain=-pi/2.5:pi/2.5,
	z buffer=sort
	]
	({0.5 * (2.3 * y) * cos(deg(x))}, {0.5 * (2 * y) * sin(deg(x))}, {2.8 * y}); % параметрическая формула поверхности
	
	\draw[->] (axis cs:\visX, 0, 0) -- (axis cs:\xmax, 0, 0); % координатные оси
	\draw[->] (axis cs:0, \visY, 0) -- (axis cs:0, \ymax, 0);
	\draw[->] (axis cs:0, 0, \visZup) -- (axis cs:0, 0, \zmax);
	
	\draw[opacity=0.25, color=black] (axis cs:0, 0, \visZdown) -- (axis cs:0, 0, \visZup); % часть оси Oz, скрытая графиком
	
	\draw[opacity=1, color=black] (axis cs:0, 0, \visZdown) -- (axis cs:0, 0, \zmin);
	
	\node at (0.4, 0, -0.25) [opacity=1] {$0$}; % нуль на оси координат
	
	\end{axis}
	
	\end{tikzpicture}
	
\end{document}
