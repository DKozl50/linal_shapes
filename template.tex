\documentclass[border=2pt]{standalone}
% \documentclass{article}
\usepackage{tikz}
\usepackage{pgfplots,tikz-3dplot}
\pgfplotsset{compat=1.17}
\makeatother
\begin{document}
\pgfplotsset{
    colormap={cmap_out}{
        color(0cm)=(blue!30!white) % колдую с оттенками для большей красоты
        color(1cm)=(red!75!white)
    }, 
}
\begin{tikzpicture}
    \def \xmax {4}
    \def \xmin {-4}
    \def \ymax {4}
    \def \ymin {-5}
    \def \zmax {5}
    \def \zmin {-5}
    \begin{axis}[
        width=15cm,
        view={145}{20},  % угол обзора 
        %
        % рисуем координатные оси, по которым вырезается картинка,
        % ставятся указатели направлений
        axis equal image,
        axis lines=center,
        xlabel={$x$},
        ylabel={$y$},
        zlabel={$z$},
        xmax=\xmax, xmin=\xmin,
        ymax=\ymax, ymin=\ymin,
        zmax=\zmax, zmin=\zmin,
        ticks=none,
        %
        axis line style={draw=none}, % но ось не отображаем (см ниже)
        colormap name=cmap_out,             % разные оттенки снаружи 
        mesh/interior colormap={cmap_in}{   % и внутри
            color(0cm)=(blue!70!black)      
            color(1cm)=(red!75!black)},
    ]
    \def \visX {1.5}   % здесь ось вылезает из графика 
    \def \visY {1}   % здесь ось вылезает из графика 
    \def \visZ {2.13}   % здесь ось вылезает из графика 
    \draw (axis cs:\xmin, 0, 0) -- (axis cs:\visX, 0, 0);      % части осей, которые        
    \draw (axis cs:0, \ymin, 0) -- (axis cs:0, \visY, 0);      % будут за 
    \draw (axis cs:0, 0, \zmin) -- (axis cs:0, 0, \visZ);    % или внутри графика
    \addplot3[                                          % сам график
        fill opacity=0.7, draw opacity=0.5,     % прозрачности ячеек и обводки ячеек
        surf,                                   
        shader=flat,                            
        samples=60,                             % количество точек. 60 - примерно максимум тикза
        domain=0:2*pi,y domain=-pi/2.5:pi/2.5,      % приходится работать с параметрическим    
        z buffer=sort]                          % уравнением. отсюда такая формула 
        ({(1.5*y*y+1.5)*cos(deg(x))}, {(y*y+1)*sin(deg(x))}, {2.5*y}); % x, y здесь параметры, а не коорд.
    \draw[->] (axis cs:\visX, 0, 0) -- (axis cs:\xmax, 0, 0);   % части осей, торчащие 
    \draw[->] (axis cs:0, \visY, 0) -- (axis cs:0, \ymax, 0);   % из графика в наблюдателя
    \draw[->] (axis cs:0, 0, \visZ) -- (axis cs:0, 0, \zmax);   % здесь пришлось подбирать вручную
    \end{axis}
\end{tikzpicture}
\end{document}
