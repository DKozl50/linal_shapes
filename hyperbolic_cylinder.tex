\documentclass[border=2pt]{standalone}
% \documentclass{article}
\usepackage{tikz}
\usepackage{pgfplots,tikz-3dplot}
\pgfplotsset{compat=1.17}
\makeatother
\begin{document}
\pgfplotsset{
    colormap={cmap_in}{
        color(0cm)=(blue!15!white)
        color(1cm)=(red)
    }, 
    colormap={cmap_in2}{
        color(0cm)=(blue!85!white)
        color(1cm)=(red)
    }, 
    colormap={cmap_out}{   
        color(0cm)=(blue!90!black)      
        color(1cm)=(red!85!black)
    },
}
\begin{tikzpicture}
    \def \xmax {5}
    \def \xmin {-5}
    \def \ymax {7}
    \def \ymin {-7}
    \def \zmax {4.5}
    \def \zmin {-4.5}
    \begin{axis}[
        width=15cm,
        view={170}{28},  
        %
        % рисуем координатные оси, по которым вырезается картинка,
        % ставятся указатели направлений
        axis equal image,
        axis lines=center,
        xlabel={$x$},
        ylabel={$y$},
        zlabel={$z$},
        xmax=\xmax, xmin=\xmin,
        ymax=\ymax, ymin=\ymin,
        zmax=\zmax, zmin=\zmin,
        ticks=none,
        %
        axis line style={draw=none}, 
        samples=20,
    ]
    \def \visXright {-4.68}
    \def \visXenter {-0.5}
    \def \visXleft {3.3}
    \def \visY {0}       

    % Части осей за графиком или внутри 
    \draw (axis cs:\visXleft, 0, 0) -- (axis cs:\visXenter, 0, 0);   
    \draw (axis cs:0, \ymin, 0) -- (axis cs:0, \visY, 0);   
    \draw (axis cs:\visXright, 0, 0) -- (axis cs:\xmin, 0, 0);   

    % сам график
    \def \xmult {0.92}
    \addplot3[                                  
        fill opacity=0.6, draw opacity=0.2,     
        surf,                                   
        shader=flat,                            
        colormap name=cmap_in,
        mesh/interior colormap name=cmap_out,
        domain=-\xmult*pi/2+pi:\xmult*pi/2+pi,y domain=-3:3,  
        z buffer=sort]                          
        ({0.5*sec(deg(x))}, {0.5*tan(deg(x))}, {y});
    \addplot3[                                  
        fill opacity=0.6, draw opacity=0.2,     
        surf,                                   
        shader=flat,                            
        colormap name=cmap_out,
        mesh/interior colormap name=cmap_in2,
        domain=-\xmult*pi/2:\xmult*pi/2,y domain=-3:3,  
        z buffer=sort]                          
        ({0.5*sec(deg(x))}, {0.5*tan(deg(x))}, {y});

    % Некоторые видимые части осей на наблюдателя 
    \draw[->] (axis cs:\visXleft, 0, 0) -- (axis cs:\xmax, 0, 0);   
    \draw[->] (axis cs:0, \visY, 0) -- (axis cs:0, \ymax, 0);   
    \draw[->] (axis cs:0, 0, \zmin) -- (axis cs:0, 0, \zmax);   

    % ось X внутри графика
    \draw[opacity=0.5] (axis cs:\visXenter, 0, 0) -- (axis cs:\visXright, 0, 0);   

    % нолик
    \node at (0.2, 0, -0.25) [opacity=1] {$0$};  
    \end{axis}
\end{tikzpicture}
\end{document}
