\documentclass[border=2pt]{standalone}

\usepackage{tikz}
\usepackage{pgfplots,tikz-3dplot}
\pgfplotsset{compat=1.16}

\makeatother

\begin{document}
\pgfplotsset{
    colormap={cmap_in}{
        color(0cm)=(blue!15!white)
        color(1cm)=(red)
    },
}


    

\begin{tikzpicture}
    \def \xmax {5}
    \def \xmin {-5}
    \def \ymax {5}
    \def \ymin {-5}
    \def \zmax {5}
    \def \zmin {-5.5}
    \begin{axis}[
        width=15cm,
        view={145}{20},  % угол обзора 
        %
        % рисуем координатные оси, по которым вырезается картинка,
        % ставятся указатели направлений
        axis equal image,
        axis lines=center,
        xlabel={$x$},
        ylabel={$y$},
        zlabel={$z$},
        xmax=\xmax, xmin=\xmin,
        ymax=\ymax, ymin=\ymin,
        zmax=\zmax, zmin=\zmin,
        ticks=none,
        axis line style={draw=none},
        colormap name=cmap_in,
        mesh/interior colormap={cmap_out}{
             color(0cm)=(blue!90!black)      
        color(1cm)=(red!85!black),}
    ]
    \def \visX {2}
    \def \visY {2}
    \def \visZup {2.9}
    \def \visZdown {-4.2}
    \draw (axis cs:\xmin, 0, 0) -- (axis cs:\visX, 0, 0);   
    \draw (axis cs:0, \ymin, 0) -- (axis cs:0, \visY, 0);   
    \draw (axis cs:0, 0, \zmin) -- (axis cs:0, 0, \visZdown);   

    \addplot3[
       fill opacity=0.5, draw opacity=0.3,
       surf,
       shader=flat,
       samples=60,
       domain=0:2*pi,y domain=-pi/2.5:pi/2.5,
       z buffer=sort]
       ({2  * cos(deg(x))}, {1.7 * sin(deg(x))}, {2.8 * y});
        
    \draw[->] (axis cs:\visX, 0, 0) -- (axis cs:\xmax, 0, 0);   
    \draw[->] (axis cs:0, \visY, 0) -- (axis cs:0, \ymax, 0);   
    \draw[->] (axis cs:0, 0, \visZup) -- (axis cs:0, 0, \zmax);   % здесь пришлось подбирать вручную




      \draw[opacity=0.25, color=black] (axis cs:0, 0, \visZdown) -- (axis cs:0, 0, \visZup);  
    \draw[opacity=1, color=black] (axis cs:0, 0, \visZdown) -- (axis cs:0, 0, \zmin);
    \node at (0.4, 0, -0.25) [opacity=0.5] {$0$};  % тоже подбор координат и прозрачности
    \end{axis}
\end{tikzpicture}

\end{document}
