\documentclass[border=2pt]{standalone}
% \documentclass{article}
\usepackage{tikz}
\usepackage{pgfplots,tikz-3dplot}
\pgfplotsset{compat=1.17}
\makeatother
\begin{document}
\pgfplotsset{
    colormap={cmap_in}{
         color(0cm)=(violet!70)      
        color(1cm)=(red!20)
    }, 
}
\begin{tikzpicture}
    \def \xmax {4}
    \def \xmin {-4}
    \def \ymax {4}
    \def \ymin {-3}
    \def \zmax {3}
    \def \zmin {-4}
    \begin{axis}[
        width=15cm,
        view={155}{30},  % угол обзора 
        %
        % рисуем координатные оси, по которым вырезается картинка,
        % ставятся указатели направлений
        axis equal image,
        axis lines=center,
        xlabel={$x$},
        ylabel={$y$},
        zlabel={$z$},
        xmax=\xmax, xmin=\xmin,
        ymax=\ymax, ymin=\ymin,
        zmax=\zmax, zmin=\zmin,
        ticks=none,
        %
        axis line style={draw=none}, % но ось не отображаем (см ниже)
        colormap name=cmap_in,             % разные оттенки внутри
        mesh/interior colormap={cmap_out}{   % и снаружи
            color(0cm)=(blue!80)  ,    
          color(1 cm)=(cyan!25)},
        samples=40,
    ]
    \def \visX {2.75}          % здесь ось вылезает из графика 
    \def \visZup {0}     % здесь ось вылезает из графика 
    \def \visZdown {-3.4}   % а здесь входит в него снизу

    % Части осей за графиком или внутри 
    \draw (axis cs:0, 0, \zmin) -- (axis cs:0, 0, \visZdown);   

    % сам график
    \addplot3[                                  
        fill opacity=0.6, draw opacity=0.2,     % прозрачности ячеек и обводки ячеек
        surf,                                   
        shader=flat,                            
        domain=-1.5:1.5,y domain=-3:3,  % приходится работать с параметрическим    
        z buffer=sort]                          % уравнением. отсюда такая формула 
        ({2.5*x}, {y}, {x^2 - y^2}/5);

    % Некоторые видимые части осей на наблюдателя 
    \draw[->] (axis cs:\visX, 0, 0) -- (axis cs:\xmax, 0, 0);   
    \draw[->] (axis cs:0, \ymin, 0) -- (axis cs:0, \ymax, 0);   
    \draw[->] (axis cs:0, 0, \visZup) -- (axis cs:0, 0, \zmax);   % здесь пришлось подбирать вручную

    % внутри графика
    \draw[opacity=0.35, color=black] (axis cs:\xmin, 0, 0) -- (axis cs:\visX, 0, 0);   
    \draw[opacity=0.35, color=black] (axis cs:0, 0, \visZdown) -- (axis cs:0, 0, \visZup);  

    % нолик
    \node at (0.2, 0, -0.25) [opacity=0.5] {$0$};  % тоже подбор координат и прозрачности
    \end{axis}
\end{tikzpicture}
\end{document}
