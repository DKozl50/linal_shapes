\documentclass[border=2pt]{standalone}

\usepackage{tikz}
\usepackage{pgfplots,tikz-3dplot}
\pgfplotsset{compat=1.16}

\makeatother

\begin{document}
\pgfplotsset{
    colormap={cmap_in}{
        color(0cm)=(blue!15!white)
        color(1cm)=(red)
    },
}

\begin{tikzpicture}
    \def \xmax {2}
    \def \xmin {-1.7}
    \def \ymax {2}
    \def \ymin {-2}
    \def \zmax {4.5}
    \def \zmin {-1}
    \begin{axis}[
        width=15cm,
        view={145}{20},  % угол обзора 
        %
        % рисуем координатные оси, по которым вырезается картинка,
        % ставятся указатели направлений
        axis equal image,
        axis lines=center,
        xlabel={$x$},
        ylabel={$y$},
        zlabel={$z$},
        xmax=\xmax, xmin=\xmin,
        ymax=\ymax, ymin=\ymin,
        zmax=\zmax, zmin=\zmin,
        ticks=none,
        %
        axis line style={draw=none}, % но ось не отображаем (см ниже)
        colormap name=cmap_in,             % разные оттенки внутри
        mesh/interior colormap={cmap_out}{   % и снаружи
              color(0cm)=(blue!70!black)      
            color(1cm)=(red!75!black)},
    ]
    \def \visX {1}          % здесь ось вылезает из графика 
    \def \visY {2/3}        % здесь ось вылезает из графика 
    \def \visZup {3.05}     % здесь ось вылезает из графика 
    \def \visZdown {-3.83}   % а здесь входит в него снизу

    % Части осей за графиком или внутри 
    \draw (axis cs:\xmin, 0, 0) -- (axis cs:\visX, 0, 0);   
    \draw (axis cs:0, \ymin, 0) -- (axis cs:0, \visY, 0);   
 %   \draw (axis cs:0, 0, \zmin) -- (axis cs:0, 0, \visZdown);   


  
      \addplot3[                                  
      fill opacity=0.5, draw opacity=0.2,     % прозрачности ячеек и обводки ячеек
       surf,                                   
       shader=flat,                            
    samples=60,                             % количество точек. 60 - примерно максимум тикза
       domain=-1:0,y domain=0:2*pi,  % приходится работать с параметрическим    
      z buffer=sort]                          % уравнением. отсюда такая формула 
       (({ 0.25 * sqrt(x^2 - 25 * x) * cos(deg(y))}, { 0.25 * sqrt( x^2 - 25 * x) * sin(deg(y))}, -3.5 * x  ); 
      
      
      
      
      
      
      
        
    % Некоторые видимые части осей на наблюдателя 
    \draw[->] (axis cs:\visX, 0, 0) -- (axis cs:\xmax, 0, 0);   
    \draw[->] (axis cs:0, \visY, 0) -- (axis cs:0, \ymax, 0);   
    \draw[->] (axis cs:0, 0, \visZup) -- (axis cs:0, 0, \zmax);   % здесь пришлось подбирать вручную

    % ось Z внутри графика
   % \draw[opacity=1, color=black] (axis cs:0, 0, -1) -- (axis cs:0, 0, 1);  


      \draw[opacity=0.25, color=black] (axis cs:0, 0, 0) -- (axis cs:0, 0, \visZup);  
    \draw[opacity=1, color=black] (axis cs:0, 0, 0) -- (axis cs:0, 0, \zmin);
    % нолик
    \node at (0.2, 0, -0.25) [opacity=1] {$0$};  % тоже подбор координат и прозрачности
    \end{axis}
\end{tikzpicture}

\end{document}
